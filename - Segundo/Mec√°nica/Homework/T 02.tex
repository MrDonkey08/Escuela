% Options for packages loaded elsewhere
\PassOptionsToPackage{unicode}{hyperref}
\PassOptionsToPackage{hyphens}{url}
%
\documentclass[
]{article}
\usepackage{amsmath,amssymb}
\usepackage{lmodern}
\usepackage{iftex}
\ifPDFTeX
  \usepackage[T1]{fontenc}
  \usepackage[utf8]{inputenc}
  \usepackage{textcomp} % provide euro and other symbols
\else % if luatex or xetex
  \usepackage{unicode-math}
  \defaultfontfeatures{Scale=MatchLowercase}
  \defaultfontfeatures[\rmfamily]{Ligatures=TeX,Scale=1}
\fi
% Use upquote if available, for straight quotes in verbatim environments
\IfFileExists{upquote.sty}{\usepackage{upquote}}{}
\IfFileExists{microtype.sty}{% use microtype if available
  \usepackage[]{microtype}
  \UseMicrotypeSet[protrusion]{basicmath} % disable protrusion for tt fonts
}{}
\makeatletter
\@ifundefined{KOMAClassName}{% if non-KOMA class
  \IfFileExists{parskip.sty}{%
    \usepackage{parskip}
  }{% else
    \setlength{\parindent}{0pt}
    \setlength{\parskip}{6pt plus 2pt minus 1pt}}
}{% if KOMA class
  \KOMAoptions{parskip=half}}
\makeatother
\usepackage{xcolor}
\usepackage{graphicx}
\makeatletter
\def\maxwidth{\ifdim\Gin@nat@width>\linewidth\linewidth\else\Gin@nat@width\fi}
\def\maxheight{\ifdim\Gin@nat@height>\textheight\textheight\else\Gin@nat@height\fi}
\makeatother
% Scale images if necessary, so that they will not overflow the page
% margins by default, and it is still possible to overwrite the defaults
% using explicit options in \includegraphics[width, height, ...]{}
\setkeys{Gin}{width=\maxwidth,height=\maxheight,keepaspectratio}
% Set default figure placement to htbp
\makeatletter
\def\fps@figure{htbp}
\makeatother
\setlength{\emergencystretch}{3em} % prevent overfull lines
\providecommand{\tightlist}{%
  \setlength{\itemsep}{0pt}\setlength{\parskip}{0pt}}
\setcounter{secnumdepth}{-\maxdimen} % remove section numbering
\ifLuaTeX
  \usepackage{selnolig}  % disable illegal ligatures
\fi
\IfFileExists{bookmark.sty}{\usepackage{bookmark}}{\usepackage{hyperref}}
\IfFileExists{xurl.sty}{\usepackage{xurl}}{} % add URL line breaks if available
\urlstyle{same} % disable monospaced font for URLs
\hypersetup{
  pdftitle={T 02},
  hidelinks,
  pdfcreator={LaTeX via pandoc}}

\title{T 02}
\author{}
\date{}

\begin{document}
\maketitle

\hypertarget{mecuxe1nica}{%
\section{Mecánica}\label{mecuxe1nica}}

\hypertarget{tarea-02}{%
\subsection{Tarea 02}\label{tarea-02}}

\hypertarget{cinemuxe1tica-de-la-partuxedcula}{%
\subsection{Cinemática de la
Partícula}\label{cinemuxe1tica-de-la-partuxedcula}}

\hypertarget{fecha-de-entrega-domingo-25-de-septiembre-de-2022}{%
\subsubsection{Fecha de Entrega: Domingo 25 de septiembre de
2022}\label{fecha-de-entrega-domingo-25-de-septiembre-de-2022}}

\hypertarget{nombre-alan-yahir-juarez-rubio}{%
\subsubsection{Nombre: Alan Yahir Juarez
Rubio}\label{nombre-alan-yahir-juarez-rubio}}

\begin{center}\rule{0.5\linewidth}{0.5pt}\end{center}

\textbf{Instrucciones.} -\/- Lea con atención los problemas y resuelva.
\textbf{Ponga primero los resultados con sus respectivas unidades (si no
están claras las respuestas no se tomará en cuenta el ejercicio, además
ponga las respuestas en orden)} y después anexe su procedimiento.

\begin{center}\rule{0.5\linewidth}{0.5pt}\end{center}

\textbf{1.-} Una bola para demolición de 75.0 kg cuelga de una cadena
uniforme de uso pesado, cuya masa es de 26.0 kg. a) Calcule las
tensiones máxima y mínima en la cadena. b) ¿Cuál es la tensión en un
punto a tres cuartos de distancia hacia arriba desde la parte inferior
de la cadena?

\textbf{Datos}\textbackslash{}
{\(\text{m}_{\text{A}} = 75\text{~kg}\)}\textbackslash{}
{\(\text{m}_{\text{b}} = 26\text{~kg}\)}\\
{\(\text{a} = 0\)}

\textbf{Fórmulas}\\
{\(\sum\overset{\rightarrow}{\text{F}} = \text{m} \cdot \text{a}\)}\\
{\(\text{T} = \text{m} \cdot \text{g}\)}

\textbf{Procedimiento inciso a}\\
{\(\text{T}_{\text{max}} = (75\text{~kg} + 26\text{~kg}) \cdot 9.81\text{~m/s}^{2}\)}\\
{\(\text{T}_{\text{max}} = 990.81\text{~kg} \cdot \text{m/s}^{2} = 990.81\text{~N}\)}

{\(\text{T}_{\text{min}} = 75\text{~kg} \cdot 9.81\text{~m/s}^{2}\)}\\
{\(\text{T}_{\text{min}} = 735.75\text{~kg} \cdot \text{m/s}^{2} = 735.75\text{~N}\)}

\textbf{Procedimiento inciso b}\\
{\(\text{T} = \left\lbrack 75\text{~kg} + \frac{3}{4}\left( 26\text{~kg} \right) \right\rbrack 9.81\text{~m/s}^{2}\)}

{\(\text{T} = 927.045\text{~kg} \cdot \text{m/s}^{2} = 806.8725\text{~N}\)}

\textbf{Respuestas inciso a}\\
{\(\text{T}_{\text{max}} = 990.81\text{~kg} \cdot \text{m/s}^{2} = 990.81\text{~N}\)}\\
{\(\text{T}_{\text{min}} = 735.75\text{~kg} \cdot \text{m/s}^{2} = 735.75\text{~N}\)}

\textbf{Respuesta inciso b}\\
{\(\text{T} = 927.045\text{~kg} \cdot \text{m/s}^{2} = 927.045\text{~N}\)}

\begin{center}\rule{0.5\linewidth}{0.5pt}\end{center}

\textbf{2.-} Se tira horizontalmente de tres trineos sobre hielo
horizontal sin fricción, usando cuerdas horizontales (como se muestra en
la figura). El tirón es horizontal y de 125 N de magnitud. Obtenga a) la
aceleración del sistema, y b) la tensión en las cuerdas A y B.

\includegraphics{https://lh4.googleusercontent.com/24m0FTziRn70JID7i6NPH7hthiMk9L0L7OON6Pik6j8kkS9iPaai0WYb5lsqYu4vgWBSgmnsb4UprDk-_cn-_rzY_kxvRrsCbnkPoPxS-TIzY6wXxvAmvZGZ2itq_zsroG81IefsxyIKZgPuD9jfTaLE9DpxqYLM7oaIF-CzziCbaw7N6Fy2OzlqNQ}

\textbf{Datos}

\textbf{Fórmulas}\textbackslash{}\\
{\(\overset{\rightarrow}{\text{a}} = \frac{\sum\overset{\rightarrow}{\text{F}}}{\text{m}}\)}\textbackslash{}
{\(\overset{\rightarrow}{\text{R}} = \sum\overset{\rightarrow}{\text{F}} = \text{m} \cdot \text{a}\)}

\textbf{Procedimiento inciso a}\\
{\(\overset{\rightarrow}{\text{a}} = \frac{125\text{~N}}{(30 + 20 + 10)\text{~kg}}\)}

{\(\overset{\rightarrow}{\text{a}} = \frac{125\text{~kg} \cdot \text{m/s}^{2}}{60\text{~kg}} \approx 2.0833\text{~m/s}^{2}\)}

\textbf{Procedimiento inciso b}\\
{\(\text{T}_{\text{A}} \approx \lbrack(30 + 20)\text{~kg}\rbrack(2.0833\text{~m/s}^{2}) = 104.1650\text{~N}\)}\\
{\(\text{T}_{\text{B}} \approx (30\text{~kg})(2.0833\text{~m/s}^{2}) = 62.499\text{~N}\)}

\textbf{Respuesta inciso A}\\
{\(\overset{\rightarrow}{\text{a}} \approx 2.0833\text{~m/s}^{2}\)}

\textbf{Respuestas inciso B}\\
{\(\text{T}_{\text{A}} \approx 104.1650\text{~N}\)}\\
{\(\text{T}_{\text{B}} \approx 62.499\text{~N}\)}

\begin{center}\rule{0.5\linewidth}{0.5pt}\end{center}

\textbf{3.-} Una velocista de alto rendimiento puede arrancar del bloque
de salida con una aceleración casi horizontal de magnitud 15
m/s{\(^{2}\)}. ¿Qué fuerza horizontal debe aplicar una corredora de 55
kg al bloque de salida al inicio para producir esta aceleración? ¿Qué
cuerpo ejerce la fuerza que impulsa a la corredora: el bloque de salida
o ella misma?

\textbf{Datos}\\
{\(\text{a} = 15\text{~m/s}^{2}\)}\\
{\(\text{m} = 55\text{~kg}\)}

\textbf{Fórmulas}\\
{\(\overset{\rightarrow}{\text{R}} = \sum\overset{\rightarrow}{\text{F}} = \text{m} \cdot \text{a}\)}

\textbf{Procedimiento inciso 1}\\
{\(\text{F} = 15\text{~m/s}^{2} \cdot 55\text{~kg} = 825\text{~N}\)}

\textbf{Repuesta pregunta 1}\\
{\(\text{F} = 825\text{~N}\)}

\textbf{Respuesta pregunta 2}\\
Ella misma

\begin{center}\rule{0.5\linewidth}{0.5pt}\end{center}

\textbf{4.-} Un avión Boeing 777 vuela horizontalmente con una rapidez
de 850 km/h (236 m/s), cuando se le cae un motor. Sin tener en cuenta la
resistencia del aire, el motor tarda 51 s en llegar al suelo. a) ¿A qué
altitud vuela el avión? b) ¿Qué distancia horizontal recorre el motor
mientras cae? c) Si el avión siguiera volando como si nada hubiera
pasado, ¿dónde estaría el motor, en relación con el avión, cuando llega
al suelo? d) Grafique la posición del motor desde que se cae del avión
hasta el suelo.

\textbf{Datos}\\
{\(\text{v} = 236\text{~m/s}\)}\\
{\(\text{t} = 51\text{~s}\)}\\
{\(\text{v}_{0_{y}} = 0\)}

\textbf{Fórmulas}\\
{\(\text{h} = \text{v}_{0_{y}}t + \frac{gt^{2}}{2}\)}\textbackslash{}\\
{\(x = x_{0} + \text{v}_{0_{x}}\text{t} + \frac{1}{2}\text{a}_{x}\text{t}^{2}\)}\\
{\(x = \text{d} \cdot \text{t}\)}

\textbf{Procedimiento inciso a}\\
{\(\text{h} = \frac{(9.81\text{~m/s}^{2})(51\text{~s})^{2}}{2}\)}

{\(\text{h} = \frac{(9.81\text{~m}/\text{s}^{2})(2601\text{~s}^{2})}{2} = 12,757.905\text{~m}\)}

\textbf{Procedimiento inciso b}\\
{\(v = \frac{850\text{~km}}{\text{h}} \cdot \frac{1000\text{~m}}{3600\text{~s}} \approx 236\text{~m/s}\)}\textbackslash{}\\
{\(x \approx (236\text{~m/s})(51\text{s}) = 12,036\text{~m}\)}

\textbf{Respuesta inciso a}\\
{\(\text{h} = 12,757.905\text{~m}\)}

\textbf{Resupesta inciso b}\\
{\(x \approx 12,036\text{~m}\)}

\textbf{Respuesta inciso c}\\
{\(r_{x} = 0\leftarrow\)} posición {\(x\)} del motor respecto al avión\\
{\(r_{y} = - 12,757.905\text{~m}\)} posición {\(y\)} del motor respecto
al avión

\begin{center}\rule{0.5\linewidth}{0.5pt}\end{center}

\textbf{5.-} Un balón es arrojado directamente hacia arriba. En el punto
más alto de su trayectoria ¿cuánto vale su aceleración? Si la rapidez
inicial del balón fue de 20 m/s, ¿cuánto tarda en llegar a su punto más
alto?

\textbf{Datos}\\
{\(\text{v}_{0} = 20\text{~m/s}\)}

\textbf{Fórmulas}\\
{\(\text{v}_{y} = \text{v}_{0_{y}} + \text{a}_{y}\text{t}\)}

\textbf{Procedimiento pregunta 2}\\
{\(0 = 20\text{~m/s} + ( - 9.81\text{~m/}\text{s}^{2})\ \text{t}\)}\\
{\(- 20\text{~m/s} = ( - 9.81\text{~m/}\text{s}^{2})\ \text{t}\)}\\
{\(\text{t} = - \frac{20\text{~m/s}}{9.81\text{~m}/\text{s}^{2}} \approx 2.039\text{~s}\)}

\textbf{Respuesta pregunta 1}\\
Al llegar a su altura máxima, su aceleración comienza a ser la de la
gravedad, pero negativa, es decir,
{\(\text{a} = - 9.81\text{~m/s}^{2}\)}

\textbf{Respuesta pregunta 2}\\
{\(\text{t} \approx 2.039\text{~s}\)}

\begin{center}\rule{0.5\linewidth}{0.5pt}\end{center}

\textbf{6.-} Rubén y Mangel ven desde un balcón de 20 m de altura una
alberca abajo; no exactamente abajo, sino a 5 m del pie de su edificio.
Se preguntan con qué rapidez deben saltar horizontalmente para caer en
la alberca. ¿Cuál es la respuesta?

\textbf{Datos}\\
{\(y = 20\text{~m}\)}\textbackslash{} {\(\text{d} = 5\text{~m}\)}

\textbf{Fórmulas}\\
{\(\text{v} = \frac{\text{d}}{\text{t}}\)}\textbackslash{}
{\(y = y_{0} + \text{v}_{0_{y}}\text{t} + \frac{1}{2}\text{a}_{y}\text{t}^{2}\)}

\textbf{Procedimiento tiempo}\\
{\(y = y_{0} + \text{v}_{0_{y}}\text{t} + \frac{1}{2}\text{a}_{y}\text{t}^{2}\)}

{\(20\text{~m} = 0 + 0\text{t} + \frac{1}{2}(9.81\text{~m/s})\ \text{t}^{2}\)}\\
{\(\frac{1}{2}(9.81\text{~m/s})\ \text{t}^{2} = 20\text{~m}\)}

{\(\text{t}^{2} = \frac{20\text{~m}}{4.905\text{~m/s}}\)}

{\(\text{t} = \pm \sqrt{\frac{20\text{~m}}{4.905\text{~m}/\text{s}}} \approx \pm \sqrt{4.077\text{~s}} \approx \pm 2.019\text{~s}\)}

{\(\text{t}_{1} \approx 2.019\text{~s}\)}\\
{\(t_{2} \approx - 2.019\text{~s}\)}

\textbf{Procedimiento rapidez}\\
{\(\text{v} \approx \frac{5\text{~m}}{2.019\text{~s}} \approx 2.4765\text{~m/s}\)}

\textbf{Respuesta}\\
{\(\text{v} \approx 2.4765\text{~m/s}\)}

\begin{center}\rule{0.5\linewidth}{0.5pt}\end{center}

\textbf{7.-} Se dispara una bala de cañón con una velocidad inicial de
141 m/s a un ángulo de 45°. Describe una trayectoria parabólica que hace
blanco en un globo, en la cúspide de su trayectoria. Sin tener en cuenta
la resistencia del aire, ¿qué rapidez tiene la bala al dar en el globo?

\textbf{Datos}\\
{\(\text{v} = 141\text{~m/s}\)}\\
{\(\alpha = 45\)}

**Fórmulas\\
{\(\text{v}_{x} = \text{v}_{0} \cdot \cos\alpha\)}\\
{\(\text{v}_{y} = \text{v}_{0} \cdot \sin\alpha\)}

**Procedimiento rapidez vector x\\
{\(\text{v}_{x} = 141\text{~m/s} \cdot \cos\ (45)\)}\\
{\(\text{v}_{x} \approx 141\text{~m/s} \cdot 0.7071 = 99.7011\text{~m/s}\)}

**Procedimiento rapidez en y max\\
{\(\text{v} = \sqrt{(99.7011\text{~m/s})^{2} + (0\text{~m/s})^{2}} = 99.7011\text{~m/s}\)}

\textbf{Respuesta}\\
{\(\text{v} = 99.7011\text{~m/s}\)}

\begin{center}\rule{0.5\linewidth}{0.5pt}\end{center}

\textbf{8.-} Máquina de Atwood. Una carga de 15.0 kg de ladrillos cuelga
del extremo de una cuerda que pasa por una polea pequeña sin fricción y
tiene un contrapeso de 28.0 kg en el otro extremo, como se muestra en la
figura. El sistema se libera del reposo. a) Dibuje dos diagramas de
cuerpo libre, uno para la carga de ladrillos y otro para el contrapeso.
b) ¿Qué magnitud tiene la aceleración hacia arriba de la carga de
ladrillos? c) ¿Qué tensión hay en la cuerda mientras la carga se mueve?
Compare esa tensión con el peso de la carga de ladrillos y con el del
contrapeso.

\includegraphics{https://lh3.googleusercontent.com/xav3K33uOWc72z_0W09Mg_T1FxtI3xCXqxe8oJUCgU5bocsDedWXJUNNd01rl6tczK32us_keh7KzbVKecR2LKVXCslsSykyJdZZ5RMV4v21o7FWXXT-IlHkFsHb1Rj4l2dsf7feMQrvuLjTRF0nd1l_Y5pefM8VQZCf_KBMJ7b-sfnYbzJFS-cWAQ}

\textbf{Datos}\\
{\(\text{M}_{\text{A}} = 15\text{~kg}\)}\\
{\(\text{M}_{\text{B}} = 28\text{~kg}\)}

\textbf{Procedimiento inciso a}

\textbf{Procedimiento inciso b}

\textbf{Procedimiento inciso c}

\textbf{Respuesta inciso a}

\textbf{Respuesta inciso b}

\textbf{Respuesta inciso c}

\begin{center}\rule{0.5\linewidth}{0.5pt}\end{center}

\textbf{9.-} Realice una lectura a detalle de los capítulos de 1 al 5
del libro Física Universitaria volumen 1 de los autores Hugh D. Young y
Roger A. Freedman (puede ser la edición que gusten). Una vez realizada
la lectura haga una retroalimentación escrita de lo que considera más
importante con una extensión mínima de media hoja.

T02 Física Universitaria - Resúmen capítulo 1 - 5

\end{document}
