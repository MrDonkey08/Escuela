% Options for packages loaded elsewhere
\PassOptionsToPackage{unicode}{hyperref}
\PassOptionsToPackage{hyphens}{url}
%
\documentclass[
]{article}
\usepackage{amsmath,amssymb}
\usepackage{lmodern}
\usepackage{iftex}
\ifPDFTeX
  \usepackage[T1]{fontenc}
  \usepackage[utf8]{inputenc}
  \usepackage{textcomp} % provide euro and other symbols
\else % if luatex or xetex
  \usepackage{unicode-math}
  \defaultfontfeatures{Scale=MatchLowercase}
  \defaultfontfeatures[\rmfamily]{Ligatures=TeX,Scale=1}
\fi
% Use upquote if available, for straight quotes in verbatim environments
\IfFileExists{upquote.sty}{\usepackage{upquote}}{}
\IfFileExists{microtype.sty}{% use microtype if available
  \usepackage[]{microtype}
  \UseMicrotypeSet[protrusion]{basicmath} % disable protrusion for tt fonts
}{}
\makeatletter
\@ifundefined{KOMAClassName}{% if non-KOMA class
  \IfFileExists{parskip.sty}{%
    \usepackage{parskip}
  }{% else
    \setlength{\parindent}{0pt}
    \setlength{\parskip}{6pt plus 2pt minus 1pt}}
}{% if KOMA class
  \KOMAoptions{parskip=half}}
\makeatother
\usepackage{xcolor}
\setlength{\emergencystretch}{3em} % prevent overfull lines
\providecommand{\tightlist}{%
  \setlength{\itemsep}{0pt}\setlength{\parskip}{0pt}}
\setcounter{secnumdepth}{-\maxdimen} % remove section numbering
\ifLuaTeX
  \usepackage{selnolig}  % disable illegal ligatures
\fi
\IfFileExists{bookmark.sty}{\usepackage{bookmark}}{\usepackage{hyperref}}
\IfFileExists{xurl.sty}{\usepackage{xurl}}{} % add URL line breaks if available
\urlstyle{same} % disable monospaced font for URLs
\hypersetup{
  pdftitle={2.5 Ejercicios},
  hidelinks,
  pdfcreator={LaTeX via pandoc}}

\title{2.5 Ejercicios}
\author{}
\date{}

\begin{document}
\maketitle

2.5.1.- Se lanza una pelota béisbol verticalmente hacia arriba con
rapidez inicial {\(v_{0}\)}. Si no se desprecia la resistencia del aire,
cuando la pelota vuelva a su altura inicial su rapidez será menor que
v0. Explique esto usando conceptos de energía

2.5.2.- Se tira de una caja de 10.0 kg usando un alambre horizontal en
un círculo sobre una superficie horizontal áspera, cuyo coeficiente de
fricción cinética es de 0.250. Calcule el trabajo efectuado por la
fricción durante un recorrido circular completo, si el radio es a) de
2.00 m y b) de 4.00 m. c) Con base en los resultados que acaba de
obtener, diría usted que la fricción es una fuerza conservativa o no
conservativa? Explique su respuesta.

\textbf{Datos}\\
{\(m = 10\text{~kg}\)}\\
{µ\(µ = 0.250\)}

\textbf{Fórmulas}\\
{\(\text{circunferencia~} = d = 2\pi r\)}\\
{µµ\(f_{k} = µ_{k}N = µ_{k}mg\)}\\
{\(W = f_{k} \cdot d\)}

\textbf{Procedimiento inciso a}\\
{\(d = 2\pi \cdot 2\text{m} \approx 12.57\text{~m}\)}\\
{\(f_{k} = 0.25 \cdot 10\text{~kg} \cdot 9.81\text{~m/s}^{2} = 24.525\text{~N}\)}\\
{\(W \approx 24.525\text{~N} \cdot 12.57 \approx 308.28\text{~Nm} = 308.28\text{~J}\)}

\textbf{Procedimiento inciso b}\\
{\(d = 2\pi \cdot 4\text{m} \approx 25.13\text{~m}\)}\\
{\(f_{k} = 0.25 \cdot 10\text{~kg} \cdot 9.81\text{~m/s}^{2} = 24.525\text{~N}\)}\\
{\(W \approx 24.525\text{~N} \cdot 25.13 \approx 616.31\text{~Nm} = 616.31\text{~J}\)}

\textbf{Respuesta inciso a}\\
{\(W \approx 308.28\text{~J}\)}

\textbf{Respuesta inciso b}\\
{\(W \approx 616.31\text{~J}\)}

2.5.3.- En un experimento, una de las fuerzas ejercidas sobre un protón
es F⃗ = −αx 2 îdonde α = 12 N/m2 a) ¿Cuánto trabajo efectúa cuando el
protón se desplaza sobre la recta del punto (0.10 m, 0) al punto (0.10
m, 0.40 m)? b) ¿Y sobre la recta del punto (0.10 m, 0) al punto (0.30 m,
0)? c) ¿Y sobre la recta del punto (0.30 m, 0) al punto (0.10 m, 0)? d)
¿es una fuerza conservativa? Explique su respuesta.

2.5.4.- Un libro de 0.60 kg se desliza sobre una mesa horizontal. La
fuerza de fricción cinética que actúa sobre el libro tiene una magnitud
de 1.2 N. a) ¿Cuánto trabajo realiza la fricción sobre el libro durante
un desplazamiento de 3.0 m a la izquierda? b) Ahora el libro se desliza
3.0 m a la derecha, volviendo al punto inicial. Durante este segundo
desplazamiento de 3.0 m, ¿qué trabajo efectúa la fricción sobre el
libro? c) ¿Qué trabajo total efectúa la fricción sobre el libro durante
el recorrido completo? d) Con base en su respuesta al inciso c), ¿diría
que la fuerza de fricción es conservativa o no conservativa? Explique su
respuesta

\end{document}
